\documentclass{article}
\usepackage[utf8]{inputenc}
\usepackage[catalan]{babel}
\usepackage{amsthm}
\usepackage{faktor}
\newcommand{\veq}{\rotatebox{90}{=}}
\newcommand{\eqto}[2]{\underset{\scriptstyle\overset{\mkern4mu\veq}{#2}}{#1}}
\newcommand{\norm}{\trianglelefteq}
\newcommand{\gal}[2]{Gal\left(\faktor{#1}{#2}\right)}
\usepackage{tikz}
\usetikzlibrary{positioning}
\usepackage[shortlabels]{enumitem}
\usepackage{amsfonts}
\usepackage{amssymb}
\usepackage{listings}
\usepackage{hyperref}
\usepackage{multicol}
\usepackage{natbib}
\usepackage{amsmath}
\usepackage{graphicx} 

\newtheorem{theorem}{Teorema}[section]
\newtheorem{corollary}{Corolari}[theorem]
\newtheorem{lemma}[theorem]{Lema}
\newtheorem{definition}[theorem]{Definició}
\newtheorem{proposition}[theorem]{Proposició}
\newtheorem{observation}[theorem]{Observació}
\newtheorem{example}[theorem]{Exemple}

\title{Treball de fi de grau}
\author{Josep Boncompte Moya}
\date{\today}

\begin{document}
\maketitle

\section{Objectius}
En aquest treball vull estudiar els algoritmes ray-tracer que creen imatges digitals amb un alt grau de realisme traçant raigs.
Analitzarè l'algoritme amb les diferents variants que es poden derivar, veient les ventatges i inconvenients d'aquests. 
També implementarè aquest algoritme aplicant-li optimitzacions.

\section{Principis de l'algoritme}
En aquesta secció explicarè els conceptes basics d'un Ray-tracer.

Començem explicant com veiem imatges a la vida real. Primer de tot necesitem una font d'il·luminació, la llum és essencial per
poder-hi veure. Aquesta font d'il·luminació desprén ratjos de llum que impacten contra objectes i reboten. Al rebotar canvien el
seu color. El que nosaltres percevem, pertant, són els ratjos que van a parar al noste ull.

%diagrama de bombeta, ratjos i ull.

En un Ray-tracer s'intenta simular el mateix. creem una font d'iluminació, uns objectes i un ull. La única diferencia és que en
aquest cas, els ratjos van en sentit contrari. Surten de l'ull i reboten contre els objectes.


%diagrama de bombeta, ratjos i ull.


Per cada raig que llencem, ens guardem el color dels objectes amb els que impacte, i formem un pixel amb aquest color.

\section{Reflexe raig amb superficie de la esfera}
%diagrama raig, normal i pla tangent
volem trobar els vectors $u$ i  $v$ tals que DIBUIX A MA
\section{interseccio de raig amb esfera}




\end{document}


