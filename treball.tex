\documentclass{article}
\usepackage[utf8]{inputenc}
\usepackage[catalan]{babel}
\usepackage{amsthm}
\usepackage{blindtext}
\usepackage{blindtext}
\usepackage{faktor}
\newcommand{\veq}{\rotatebox{90}{=}}
\newcommand{\eqto}[2]{\underset{\scriptstyle\overset{\mkern4mu\veq}{#2}}{#1}}
\newcommand{\norm}{\trianglelefteq}
\newcommand{\gal}[2]{Gal\left(\faktor{#1}{#2}\right)}
\usepackage{tikz}
\usetikzlibrary{positioning}
\usepackage[shortlabels]{enumitem}
\usepackage{amsfonts}
\usepackage{amssymb}
\usepackage{listings}
\usepackage{hyperref}
\usepackage{multicol}
\usepackage{natbib}
\usepackage{amsmath}
\usepackage{graphicx} 
\usepackage{imakeidx}
\newtheorem{theorem}{Teorema}[section]
\newtheorem{corollary}{Corolari}[theorem]
\newtheorem{lemma}[theorem]{Lema}
\newtheorem{definition}[theorem]{Definició}
\newtheorem{proposition}[theorem]{Proposició}
\newtheorem{observation}[theorem]{Observació}
\newtheorem{example}[theorem]{Exemple}

\title{Introducció als algoritmes de Ray casting}
\author{Josep Boncompte Moya}
\date{\today}
\makeindex
\begin{document}
\maketitle
\printindex
\section{Objectius}
En aquest treball vull estudiar els algoritmes ray casting que creen imatges digitals amb un alt grau de realisme traçant raigs.
Analitzarè l'algoritme amb les diferents variants que es poden derivar, veient les avantatges i inconvenients d'aquests. 
També implementarè aquest algoritme aplicant-li optimitzacions.
\clearpage
\section{Principis de l'algoritme}
En aquesta secció explicarè els conceptes bàsics d'un Ray casting. Comencem explicant com veiem imatges a la vida real. 

Primer de tot necesitem una font d'il·luminació, la llum és essencial per
poder-hi veure. Aquesta font d'il·luminació desprén raigs de llum que impacten contra objectes i reboten. Al rebotar canvien el
seu color. El que nosaltres percebem, per tant, són els raigs que van a parar al noste ull.

En un Ray caster s'intenta simular el mateix. Creem una font d'il·luminació, uns objectes, una quadrícula i un ull. 
En comptes de calcular tots els raigs que surten de la font d'il·luminació i quedar-nos amb els
que arriben a l'ull, calculem els raigs en sentit contrari. D'aquesta manera el process és molt més eficient.
Per tant, els raigs surten de l'ull i reboten contra els objectes i només tindràn color aquells que arribin a la font
d'il·luminació.

\begin{figure}
\centering
\label{fig:reflex}


\begin{tikzpicture}[x=0.75pt,y=0.75pt,yscale=-0.7,xscale=0.7]
%uncomment if require: \path (0,300); %set diagram left start at 0, and has height of 300

%Shape: Grid [id:dp2874209353417645] 
\draw  [draw opacity=0] (130.29,110.8) -- (105.57,205.4) -- (213.66,177.24) -- (238.37,82.64) -- cycle ; \draw  [color={rgb, 255:red, 155; green, 155; blue, 155 }  ,draw opacity=1 ] (130.03,111.76) -- (238.12,83.61)(127,123.37) -- (235.09,95.22)(123.97,134.99) -- (232.05,106.83)(120.93,146.6) -- (229.02,118.44)(117.9,158.21) -- (225.99,130.05)(114.87,169.82) -- (222.95,141.66)(111.84,181.43) -- (219.92,153.27)(108.8,193.04) -- (216.89,164.88)(105.77,204.65) -- (213.85,176.49) ; \draw  [color={rgb, 255:red, 155; green, 155; blue, 155 }  ,draw opacity=1 ] (131.39,110.51) -- (106.68,205.11)(144.7,107.04) -- (119.99,201.64)(158.01,103.58) -- (133.3,198.17)(171.32,100.11) -- (146.61,194.71)(184.63,96.64) -- (159.92,191.24)(197.94,93.18) -- (173.22,187.77)(211.24,89.71) -- (186.53,184.31)(224.55,86.24) -- (199.84,180.84)(237.86,82.78) -- (213.15,177.37) ; \draw  [color={rgb, 255:red, 155; green, 155; blue, 155 }  ,draw opacity=1 ]  ;
%Flowchart: Summing Junction [id:dp1397553415106878] 
\draw  [color={rgb, 255:red, 74; green, 144; blue, 226 }  ,draw opacity=1 ][line width=1.5]  (45,142.5) .. controls (45,136.15) and (50.15,131) .. (56.5,131) .. controls (62.85,131) and (68,136.15) .. (68,142.5) .. controls (68,148.85) and (62.85,154) .. (56.5,154) .. controls (50.15,154) and (45,148.85) .. (45,142.5) -- cycle ; \draw  [color={rgb, 255:red, 74; green, 144; blue, 226 }  ,draw opacity=1 ][line width=1.5]  (48.37,134.37) -- (64.63,150.63) ; \draw  [color={rgb, 255:red, 74; green, 144; blue, 226 }  ,draw opacity=1 ][line width=1.5]  (64.63,134.37) -- (48.37,150.63) ;
%Shape: Circle [id:dp5662574699750227] 
\draw  [fill={rgb, 255:red, 189; green, 16; blue, 224 }  ,fill opacity=1 ] (321,30) .. controls (321,16.19) and (332.19,5) .. (346,5) .. controls (359.81,5) and (371,16.19) .. (371,30) .. controls (371,43.81) and (359.81,55) .. (346,55) .. controls (332.19,55) and (321,43.81) .. (321,30) -- cycle ;
%Shape: Circle [id:dp1275608568426868] 
\draw  [fill={rgb, 255:red, 126; green, 211; blue, 33 }  ,fill opacity=1 ] (412,135) .. controls (412,112.91) and (429.91,95) .. (452,95) .. controls (474.09,95) and (492,112.91) .. (492,135) .. controls (492,157.09) and (474.09,175) .. (452,175) .. controls (429.91,175) and (412,157.09) .. (412,135) -- cycle ;
%Straight Lines [id:da3390406793374766] 
\draw    (56.5,142.5) -- (414.35,123.52) ;
\draw [shift={(416.35,123.41)}, rotate = 176.96] [color={rgb, 255:red, 0; green, 0; blue, 0 }  ][line width=0.75]    (10.93,-3.29) .. controls (6.95,-1.4) and (3.31,-0.3) .. (0,0) .. controls (3.31,0.3) and (6.95,1.4) .. (10.93,3.29)   ;
%Straight Lines [id:da9475530524890181] 
\draw    (416.35,123.41) -- (371.92,37.78) ;
\draw [shift={(371,36)}, rotate = 62.58] [color={rgb, 255:red, 0; green, 0; blue, 0 }  ][line width=0.75]    (10.93,-3.29) .. controls (6.95,-1.4) and (3.31,-0.3) .. (0,0) .. controls (3.31,0.3) and (6.95,1.4) .. (10.93,3.29)   ;

% Text Node
\draw (328,21) node [anchor=north west][inner sep=0.75pt]   [align=left] {Llum};
% Text Node
\draw (425,125) node [anchor=north west][inner sep=0.75pt]   [align=left] {Objecte};
% Text Node
\draw (45,109) node [anchor=north west][inner sep=0.75pt]   [align=left] {Ull};
% Text Node
\draw (133,59) node [anchor=north west][inner sep=0.75pt]   [align=left] {matriu d'imatge};
% Text Node
\draw (275,108) node [anchor=north west][inner sep=0.75pt]   [align=left] {Raig};


\end{tikzpicture}
   
\caption{Representació del funcionament bàsic de l'algoritme de Ray casting}
\end{figure}

La quadrícula representarà el conjunt de píxels de la imatge resultant.

Per cada píxel de la quadrícula llencem un raig a través, ens guardem el color dels objectes amb els quals impacti.
Per poder efectuar aquest algoritme ens caldrà calcular:
  \begin{enumerate}[a)]
     \item La intersecció d'un raig amb un objecte.
     \item rebot d'un raig amb una superfície.
     \item  repetir a) i b) amb un nou raig.
  \end{enumerate} 
\subsection{intersecció del raig amb objectes}
Per saber si un raig intersecta amb un objecte ho podem fer de dues formes diferents:
\subsubsection{Ray tracing}
Ray tracing és una variant dels algoritmes de ray casting que resol un sistema d'equacions per trobar la intersecció del raig amb
l'objecte. Aquesta variant és molt eficient però està limitada a superfícies que podem parametritzar o que tinguin una fórmula
senzilla.
\begin{enumerate}[a)]
   \item Començarem dibuixant esferes, ja que és l'objecte més simple.
  Podem definir un raig amb l'equació d'una recta $f(t)=dt+p$ on $d=(d_1,d_2,d_3)$ és el vector director normalitzat, i $p=(p_1,p_2,p_3)$ és
  el punt de la recta quan $t=0$. 
  Més endavant utilitzarem raigs no rectes per simular la curvatura de l'espai.
  Podem expressar la recta de la forma següent:
\begin{align*}
x=p_1+td_1 \\
y=p_2+td_2 \\
z=p_3+td_3 \\
\end{align*}
Per una esfera de radi $r$ i centre  $c=(c_1,c_2,c_3)$ podem definir-la com:
\begin{align*}
   (x-c_1)^2+(y-c_2)^2 +(z-c_3)^2 =r^2
\end{align*}
Per trobar els punts d'intersecció només cal substituir $x,y,z$ obtenint:
 \begin{align*}
   ((p_1-c_1)+td_1)^2 + ((p_2-c_2)+td_2)^2+((p_3-c_3)+td_3)^2= r^2  \iff \\
   (td_1)^2+(td_2)^2+(td_3)^2+ (p_1-c_1)^2+(p_2-c_2)^2+(p_3-c_3)^2+\\ 2(p_1-c_1)td_1+2(p_2-c_2)td_2+2(p_3-c_3)td_3 -r^2=0  \iff\\
   \|d\|^2t^2 + \|p-c\|^2+2t(<p-c,d>) -r^2=0
\end{align*}
Obtenint una equació de segon grau $at^2+bt+c=0$ on $a= \|d\|^2 =1$ per definició, $b=2(<p-c,d>)$ i  $c=\|p-c\|^2-r^2$. 

Aquesta fórmula ens donarà 0, 1 o 2 solucions, que representen el temps $t$ en el qual la recta talla l'esfera.

Perque l'algoritme tingui sentit hem d'escollir la solució positiva més petita.
% diagrama 
\item Per calcular la intersecció d'un raig amb un cub tindrem els següents conceptes presents:


   ...
\end{enumerate}

\subsubsection{Ray marching}
Ray marching és un altre variant dels algoritmes de ray casting que calcula la intersecció d'un raig de forma recursiva a partir
de la distància del raig al objecte.
El metode és el següent:
\begin{enumerate}[a)]
   \item Creem un raig $f(t)= dt+p$. amb  $d$ la direcció i  $p $ la posició inicial.
    \item  Calculem la distància $D$ minima entre p i el objecte.
       \item abançem en la direcció $d$ aquesta distància i definim  la nova posició com
          \begin{equation*}
              p'=D\cdot d+p
          \end{equation*}
       \item repetim aquest procés fins que la $D=0$ o fins que  $D= \infty$ ( en el cas que no intersectin ).
\end{enumerate}


\begin{figure}
   \centering

\begin{tikzpicture}[x=0.75pt,y=0.75pt,yscale=-0.7,xscale=0.7]
%uncomment if require: \path (0,300); %set diagram left start at 0, and has height of 300

%Straight Lines [id:da5172579435437765] 
\draw [color={rgb, 255:red, 189; green, 16; blue, 224 }  ,draw opacity=1 ][line width=1.5]    (49,129) -- (295.64,167.34) ;
\draw [shift={(298.6,167.8)}, rotate = 188.84] [color={rgb, 255:red, 189; green, 16; blue, 224 }  ,draw opacity=1 ][line width=1.5]    (14.21,-4.28) .. controls (9.04,-1.82) and (4.3,-0.39) .. (0,0) .. controls (4.3,0.39) and (9.04,1.82) .. (14.21,4.28)   ;
%Shape: Circle [id:dp8891369335784802] 
\draw  [fill={rgb, 255:red, 184; green, 233; blue, 134 }  ,fill opacity=1 ] (151.5,92.5) .. controls (151.5,72.07) and (168.07,55.5) .. (188.5,55.5) .. controls (208.93,55.5) and (225.5,72.07) .. (225.5,92.5) .. controls (225.5,112.93) and (208.93,129.5) .. (188.5,129.5) .. controls (168.07,129.5) and (151.5,112.93) .. (151.5,92.5) -- cycle ;
%Shape: Circle [id:dp28054031526846746] 
\draw  [fill={rgb, 255:red, 0; green, 0; blue, 0 }  ,fill opacity=1 ] (44.5,129) .. controls (44.5,126.51) and (46.51,124.5) .. (49,124.5) .. controls (51.49,124.5) and (53.5,126.51) .. (53.5,129) .. controls (53.5,131.49) and (51.49,133.5) .. (49,133.5) .. controls (46.51,133.5) and (44.5,131.49) .. (44.5,129) -- cycle ;
%Shape: Circle [id:dp1171865334383273] 
\draw  [fill={rgb, 255:red, 0; green, 0; blue, 0 }  ,fill opacity=1 ] (149,145) .. controls (149,142.51) and (151.01,140.5) .. (153.5,140.5) .. controls (155.99,140.5) and (158,142.51) .. (158,145) .. controls (158,147.49) and (155.99,149.5) .. (153.5,149.5) .. controls (151.01,149.5) and (149,147.49) .. (149,145) -- cycle ;
%Shape: Circle [id:dp6095132253359951] 
\draw  [line width=0.75]  (127.44,145) .. controls (127.44,130.61) and (139.11,118.94) .. (153.5,118.94) .. controls (167.89,118.94) and (179.56,130.61) .. (179.56,145) .. controls (179.56,159.39) and (167.89,171.06) .. (153.5,171.06) .. controls (139.11,171.06) and (127.44,159.39) .. (127.44,145) -- cycle ;
%Shape: Circle [id:dp8256341875574109] 
\draw  [fill={rgb, 255:red, 0; green, 0; blue, 0 }  ,fill opacity=1 ] (175.06,149.5) .. controls (175.06,147.01) and (177.08,145) .. (179.56,145) .. controls (182.05,145) and (184.06,147.01) .. (184.06,149.5) .. controls (184.06,151.99) and (182.05,154) .. (179.56,154) .. controls (177.08,154) and (175.06,151.99) .. (175.06,149.5) -- cycle ;
%Shape: Circle [id:dp4615529528953036] 
\draw  [line width=0.75]  (158.77,149.5) .. controls (158.77,138.02) and (168.08,128.71) .. (179.56,128.71) .. controls (191.04,128.71) and (200.35,138.02) .. (200.35,149.5) .. controls (200.35,160.98) and (191.04,170.29) .. (179.56,170.29) .. controls (168.08,170.29) and (158.77,160.98) .. (158.77,149.5) -- cycle ;
%Shape: Circle [id:dp8488546667039918] 
\draw  [fill={rgb, 255:red, 0; green, 0; blue, 0 }  ,fill opacity=1 ] (195.85,152.5) .. controls (195.85,150.01) and (197.87,148) .. (200.35,148) .. controls (202.84,148) and (204.85,150.01) .. (204.85,152.5) .. controls (204.85,154.99) and (202.84,157) .. (200.35,157) .. controls (197.87,157) and (195.85,154.99) .. (195.85,152.5) -- cycle ;
%Shape: Circle [id:dp8827999503916382] 
\draw  [line width=0.75]  (176.77,152.5) .. controls (176.77,139.48) and (187.33,128.92) .. (200.35,128.92) .. controls (213.38,128.92) and (223.94,139.48) .. (223.94,152.5) .. controls (223.94,165.52) and (213.38,176.08) .. (200.35,176.08) .. controls (187.33,176.08) and (176.77,165.52) .. (176.77,152.5) -- cycle ;
%Shape: Circle [id:dp28365635266662215] 
\draw  [fill={rgb, 255:red, 0; green, 0; blue, 0 }  ,fill opacity=1 ] (219.44,157) .. controls (219.44,154.51) and (221.45,152.5) .. (223.94,152.5) .. controls (226.42,152.5) and (228.44,154.51) .. (228.44,157) .. controls (228.44,159.49) and (226.42,161.5) .. (223.94,161.5) .. controls (221.45,161.5) and (219.44,159.49) .. (219.44,157) -- cycle ;
%Shape: Circle [id:dp15779047756636633] 
\draw  [line width=0.75]  (187.13,157) .. controls (187.13,136.67) and (203.61,120.19) .. (223.94,120.19) .. controls (244.26,120.19) and (260.74,136.67) .. (260.74,157) .. controls (260.74,177.33) and (244.26,193.81) .. (223.94,193.81) .. controls (203.61,193.81) and (187.13,177.33) .. (187.13,157) -- cycle ;
%Shape: Arc [id:dp1856201303120597] 
\draw  [draw opacity=0] (19.57,30.27) .. controls (30.83,26.14) and (42.98,24.01) .. (55.62,24.26) .. controls (111.89,25.4) and (156.55,73.51) .. (155.37,131.73) .. controls (154.19,189.94) and (107.62,236.2) .. (51.35,235.06) .. controls (46.71,234.97) and (42.14,234.56) .. (37.68,233.84) -- (53.49,129.66) -- cycle ; \draw   (19.57,30.27) .. controls (30.83,26.14) and (42.98,24.01) .. (55.62,24.26) .. controls (111.89,25.4) and (156.55,73.51) .. (155.37,131.73) .. controls (154.19,189.94) and (107.62,236.2) .. (51.35,235.06) .. controls (46.71,234.97) and (42.14,234.56) .. (37.68,233.84) ;
%Shape: Circle [id:dp9597561829051393] 
\draw  [fill={rgb, 255:red, 0; green, 0; blue, 0 }  ,fill opacity=1 ] (256.24,161.5) .. controls (256.24,159.01) and (258.26,157) .. (260.74,157) .. controls (263.23,157) and (265.24,159.01) .. (265.24,161.5) .. controls (265.24,163.99) and (263.23,166) .. (260.74,166) .. controls (258.26,166) and (256.24,163.99) .. (256.24,161.5) -- cycle ;
%Straight Lines [id:da42016882542000444] 
\draw    (49,129) -- (149.57,102.26) ;
\draw [shift={(151.5,101.75)}, rotate = 165.11] [color={rgb, 255:red, 0; green, 0; blue, 0 }  ][line width=0.75]    (10.93,-3.29) .. controls (6.95,-1.4) and (3.31,-0.3) .. (0,0) .. controls (3.31,0.3) and (6.95,1.4) .. (10.93,3.29)   ;
%Straight Lines [id:da8453217545764917] 
\draw [color={rgb, 255:red, 189; green, 16; blue, 224 }  ,draw opacity=1 ][line width=1.5]    (385.26,149) -- (514.67,149.65) ;
\draw [shift={(517.67,149.67)}, rotate = 180.29] [color={rgb, 255:red, 189; green, 16; blue, 224 }  ,draw opacity=1 ][line width=1.5]    (14.21,-4.28) .. controls (9.04,-1.82) and (4.3,-0.39) .. (0,0) .. controls (4.3,0.39) and (9.04,1.82) .. (14.21,4.28)   ;
%Shape: Circle [id:dp41846170941239447] 
\draw  [fill={rgb, 255:red, 184; green, 233; blue, 134 }  ,fill opacity=1 ] (487.76,112.5) .. controls (487.76,92.07) and (504.33,75.5) .. (524.76,75.5) .. controls (545.2,75.5) and (561.76,92.07) .. (561.76,112.5) .. controls (561.76,132.93) and (545.2,149.5) .. (524.76,149.5) .. controls (504.33,149.5) and (487.76,132.93) .. (487.76,112.5) -- cycle ;
%Shape: Circle [id:dp36730335072294995] 
\draw  [fill={rgb, 255:red, 0; green, 0; blue, 0 }  ,fill opacity=1 ] (380.76,149) .. controls (380.76,146.51) and (382.78,144.5) .. (385.26,144.5) .. controls (387.75,144.5) and (389.76,146.51) .. (389.76,149) .. controls (389.76,151.49) and (387.75,153.5) .. (385.26,153.5) .. controls (382.78,153.5) and (380.76,151.49) .. (380.76,149) -- cycle ;
%Shape: Circle [id:dp20206991311889755] 
\draw  [fill={rgb, 255:red, 0; green, 0; blue, 0 }  ,fill opacity=1 ] (487.76,149.4) .. controls (487.76,148.07) and (488.84,147) .. (490.16,147) .. controls (491.49,147) and (492.56,148.07) .. (492.56,149.4) .. controls (492.56,150.73) and (491.49,151.8) .. (490.16,151.8) .. controls (488.84,151.8) and (487.76,150.73) .. (487.76,149.4) -- cycle ;
%Shape: Circle [id:dp09693741355799557] 
\draw  [fill={rgb, 255:red, 0; green, 0; blue, 0 }  ,fill opacity=1 ] (501.35,149.4) .. controls (501.35,148.04) and (502.45,146.94) .. (503.81,146.94) .. controls (505.17,146.94) and (506.27,148.04) .. (506.27,149.4) .. controls (506.27,150.76) and (505.17,151.86) .. (503.81,151.86) .. controls (502.45,151.86) and (501.35,150.76) .. (501.35,149.4) -- cycle ;
%Shape: Arc [id:dp6926610813158203] 
\draw  [draw opacity=0] (354.8,43.21) .. controls (366.06,39.08) and (378.21,36.94) .. (390.85,37.2) .. controls (447.12,38.34) and (491.77,86.45) .. (490.6,144.66) .. controls (489.42,202.87) and (442.85,249.14) .. (386.58,248) .. controls (381.93,247.91) and (377.37,247.49) .. (372.9,246.78) -- (388.71,142.6) -- cycle ; \draw   (354.8,43.21) .. controls (366.06,39.08) and (378.21,36.94) .. (390.85,37.2) .. controls (447.12,38.34) and (491.77,86.45) .. (490.6,144.66) .. controls (489.42,202.87) and (442.85,249.14) .. (386.58,248) .. controls (381.93,247.91) and (377.37,247.49) .. (372.9,246.78) ;
%Straight Lines [id:da7255127358487097] 
\draw    (385.26,149) -- (485.83,122.26) ;
\draw [shift={(487.76,121.75)}, rotate = 165.11] [color={rgb, 255:red, 0; green, 0; blue, 0 }  ][line width=0.75]    (10.93,-3.29) .. controls (6.95,-1.4) and (3.31,-0.3) .. (0,0) .. controls (3.31,0.3) and (6.95,1.4) .. (10.93,3.29)   ;
%Shape: Circle [id:dp4307172386151815] 
\draw   (476.51,149.4) .. controls (476.51,141.86) and (482.62,135.75) .. (490.16,135.75) .. controls (497.7,135.75) and (503.81,141.86) .. (503.81,149.4) .. controls (503.81,156.94) and (497.7,163.05) .. (490.16,163.05) .. controls (482.62,163.05) and (476.51,156.94) .. (476.51,149.4) -- cycle ;
%Shape: Circle [id:dp21822413838175214] 
\draw   (499.22,149.4) .. controls (499.22,146.87) and (501.28,144.81) .. (503.81,144.81) .. controls (506.35,144.81) and (508.4,146.87) .. (508.4,149.4) .. controls (508.4,151.93) and (506.35,153.99) .. (503.81,153.99) .. controls (501.28,153.99) and (499.22,151.93) .. (499.22,149.4) -- cycle ;

% Text Node
\draw (98.5,118) node [anchor=north west][inner sep=0.75pt]   [align=left] {$\displaystyle \textcolor[rgb]{0.74,0.06,0.88}{\vec{r}}$};
% Text Node
\draw (37.07,129.57) node [anchor=north west][inner sep=0.75pt]   [align=left] {$\displaystyle \textcolor[rgb]{0.74,0.06,0.88}{p}$};
% Text Node
\draw (95.5,95) node [anchor=north west][inner sep=0.75pt]   [align=left] {$\displaystyle D$};
% Text Node
\draw (435.43,152) node [anchor=north west][inner sep=0.75pt]   [align=left] {$\displaystyle \textcolor[rgb]{0.74,0.06,0.88}{\vec{r}}$};
% Text Node
\draw (373.33,149.57) node [anchor=north west][inner sep=0.75pt]   [align=left] {$\displaystyle \textcolor[rgb]{0.74,0.06,0.88}{p}$};
% Text Node
\draw (431.76,115) node [anchor=north west][inner sep=0.75pt]   [align=left] {$\displaystyle D$};


\end{tikzpicture}
\caption{Calcul d'intersecció utilitzant la variant Ray marching}   
\label{fig:3}
\end{figure}
Aquesta variant de Ray casting és costosa computacionalment, però ens permet utilitzar objectes més complexos dels quals no podem
trobar algebraicament la intersecció amb el raig. 

\subsection{Reflexió d'un raig amb una superfície}
Donat un raig i una superfície que intersequen, volem trobar el vector director resultant de la reflexió.

Ens caldrà, per tant trobar el plà tangent de la superfície en el punt d'intersecció.


Sabem que trobar el pla tangent és equivalent a trobar el vector normal de la superfície en aquell punt.


En el cas de l'esfera, el vector normal és el mateix que el vector posició (si l'esfera està centada en l'origen). 
En el cas de superfícies parapetritzades $X(u,v)$ el vector normal és:
 \begin{equation*}
    N= \frac{\partial X}{\partial u} \times \frac{\partial X}{\partial v}
\end{equation*}
i en el cas general 
\begin{equation*}
   \hat{N}= \hat{B}\times \hat{T}
\end{equation*}
on $\hat{T}$ és el vector tangent unitari i $\hat{B}$ és el vector binormal unitari.
\begin{figure}[h]
\centering
\begin{tikzpicture}[x=0.75pt,y=0.75pt,yscale=-1.1,xscale=1.1]
%uncomment if require: \path (0,288); %set diagram left start at 0, and has height of 288

%Shape: Arc [id:dp13742492516777927] 
\draw  [draw opacity=0][fill={rgb, 255:red, 126; green, 211; blue, 33 }  ,fill opacity=0.3 ] (231.09,201.57) .. controls (244.37,176.19) and (273.57,158.41) .. (307.62,158.08) .. controls (336.6,157.8) and (362.31,170.22) .. (377.82,189.4) -- (308.34,232.13) -- cycle ; \draw   (231.09,201.57) .. controls (244.37,176.19) and (273.57,158.41) .. (307.62,158.08) .. controls (336.6,157.8) and (362.31,170.22) .. (377.82,189.4) ;
%Shape: Parallelogram [id:dp8175608544803742] 
\draw  [fill={rgb, 255:red, 74; green, 144; blue, 226 }  ,fill opacity=0.69 ] (281.23,146.71) -- (353.28,146.56) -- (314.77,173.29) -- (242.72,173.44) -- cycle ;
%Straight Lines [id:da29942039095263373] 
\draw [color={rgb, 255:red, 0; green, 0; blue, 0 }  ,draw opacity=1 ]   (298,160) -- (291.26,117.46) -- (290.31,111.48) ;
\draw [shift={(290,109.5)}, rotate = 81] [color={rgb, 255:red, 0; green, 0; blue, 0 }  ,draw opacity=1 ][line width=0.75]    (10.93,-3.29) .. controls (6.95,-1.4) and (3.31,-0.3) .. (0,0) .. controls (3.31,0.3) and (6.95,1.4) .. (10.93,3.29)   ;
%Straight Lines [id:da9353866239685801] 
\draw [color={rgb, 255:red, 189; green, 16; blue, 224 }  ,draw opacity=1 ]   (255,131.75) -- (296.33,158.9) ;
\draw [shift={(298,160)}, rotate = 213.3] [color={rgb, 255:red, 189; green, 16; blue, 224 }  ,draw opacity=1 ][line width=0.75]    (10.93,-3.29) .. controls (6.95,-1.4) and (3.31,-0.3) .. (0,0) .. controls (3.31,0.3) and (6.95,1.4) .. (10.93,3.29)   ;
%Straight Lines [id:da4494224552982866] 
\draw [color={rgb, 255:red, 189; green, 16; blue, 224 }  ,draw opacity=1 ] [dash pattern={on 4.5pt off 4.5pt}]  (298,160) -- (326.84,119.38) ;
\draw [shift={(328,117.75)}, rotate = 125.38] [color={rgb, 255:red, 189; green, 16; blue, 224 }  ,draw opacity=1 ][line width=0.75]    (10.93,-3.29) .. controls (6.95,-1.4) and (3.31,-0.3) .. (0,0) .. controls (3.31,0.3) and (6.95,1.4) .. (10.93,3.29)   ;
%Straight Lines [id:da3404688386636361] 
\draw [line width=0.75]  [dash pattern={on 0.84pt off 2.51pt}]  (264.5,165.25) -- (292.71,160.71) -- (296.02,160.27) ;
\draw [shift={(298,160)}, rotate = 172.3] [color={rgb, 255:red, 0; green, 0; blue, 0 }  ][line width=0.75]    (10.93,-3.29) .. controls (6.95,-1.4) and (3.31,-0.3) .. (0,0) .. controls (3.31,0.3) and (6.95,1.4) .. (10.93,3.29)   ;
%Straight Lines [id:da821390721067782] 
\draw [line width=0.75]  [dash pattern={on 0.84pt off 2.51pt}]  (255,131.75) -- (262.26,157.33) -- (263.95,163.33) ;
\draw [shift={(264.5,165.25)}, rotate = 254.17] [color={rgb, 255:red, 0; green, 0; blue, 0 }  ][line width=0.75]    (10.93,-3.29) .. controls (6.95,-1.4) and (3.31,-0.3) .. (0,0) .. controls (3.31,0.3) and (6.95,1.4) .. (10.93,3.29)   ;
%Straight Lines [id:da7295682968328961] 
\draw [line width=0.75]  [dash pattern={on 0.84pt off 2.51pt}]  (334.5,153) -- (328.36,119.72) ;
\draw [shift={(328,117.75)}, rotate = 79.55] [color={rgb, 255:red, 0; green, 0; blue, 0 }  ][line width=0.75]    (10.93,-3.29) .. controls (6.95,-1.4) and (3.31,-0.3) .. (0,0) .. controls (3.31,0.3) and (6.95,1.4) .. (10.93,3.29)   ;
%Straight Lines [id:da6660555012051472] 
\draw [line width=0.75]  [dash pattern={on 0.84pt off 2.51pt}]  (298,160) -- (332.54,153.38) ;
\draw [shift={(334.5,153)}, rotate = 169.14] [color={rgb, 255:red, 0; green, 0; blue, 0 }  ][line width=0.75]    (10.93,-3.29) .. controls (6.95,-1.4) and (3.31,-0.3) .. (0,0) .. controls (3.31,0.3) and (6.95,1.4) .. (10.93,3.29)   ;

% Text Node
\draw (245.5,113) node [anchor=north west][inner sep=0.75pt]  [color={rgb, 255:red, 189; green, 16; blue, 224 }  ,opacity=1 ]
   [align=left] {$\displaystyle \vec{R}$};
% Text Node
\draw (329,93.5) node [anchor=north west][inner sep=0.75pt]  [color={rgb, 255:red, 189; green, 16; blue, 224 }  ,opacity=1 ]
   [align=left] {$\displaystyle \overrightarrow{R'}$};
% Text Node
\draw (282.26,87.26) node [anchor=north west][inner sep=0.75pt]  [color={rgb, 255:red, 189; green, 16; blue, 224 }  ,opacity=1 ,rotate=-352.23] [align=left] {$\displaystyle \textcolor[rgb]{0.49,0.83,0.13}{\vec{\textcolor[rgb]{0.49,0.83,0.13}{N}}}$};
% Text Node
\draw (271,163) node [anchor=north west][inner sep=0.75pt] [align=left] {$\displaystyle {R_{\bot}}$};
% Text Node
\draw (244,139.37) node [anchor=north west][inner sep=0.75pt]  [color={rgb, 255:red, 189; green, 16; blue, 224 }  ,opacity=1 ,rotate=-1.53] [align=left] {$\displaystyle \textcolor[rgb]{0,0,0}{R_{\parallel}}$};


\end{tikzpicture}
\label{fig:reflex}
\caption{Reflex d'un raig amb una esfera}
\end{figure}



Seguint la terminologia del la figura ref %referencia% 
volem trobar  el vector $R'$ que descompon en 
\begin{equation*}
   R'= R_{\bot}- R_{\parallel}
\end{equation*}
tals que $R_{\parallel}$ sigui la projecció de $R$ sobre  $N$ i  $R_{\bot}$ la component de $R$ perpendicular a $N$.



El reflexe d' $R$ sobre la superfície només canvia la component $R_{\parallel}$ (de sentit). La component  $R_{\bot}$ es manté igual.


Anem doncs a calcular $R_{\parallel}$.
\begin{align} 
   R_{\parallel}=\|R\| \cos(\theta)  N \\
   \cos(\theta) = <-R , N> \\
   R_{\bot}= R-R_{\parallel}
\end{align}
on $\theta$ és l'angle que forman el vector  $-R$ i  $N$ (producte escalar).
Per tant el vector resultant és
\begin{equation*}
   R'= R-2(\cos(\theta)N)
\end{equation*}

\subsection{Repetir 2.1 i 2.2 per un nou raig}
un cop hem calculat la intersecció del raig amb l'objecte i tenim el vector de la reflexió, repetirem el procés amb el nou raig
$f(t)=dt+p$ on  $d$ és el vector resultat de la reflexió i  $p$ és el punt d'interseció. Depen de l'escena que estiguem
representant és possible que el raig reboti infinitament, per tant, haurem de posar-li un limit, el qual si s'assoleix, el color
d'aquest píxel serà negre.
\section{Component de la camara}
Explicar obertura focal.
\section{Suavitzat d'imatge, Metode de Monte Carlo}
Donada una escena amb un conjunt d'objectes, és difícil simular la il·luminació 
indirecta. Ja que cada punt $P$ d'una superfície
rep llum indirecta de tots els altres objectes. 
\begin{figure}[h]
   \centering
   
\tikzset{every picture/.style={line width=0.75pt}} %set default line width to 0.75pt        

\begin{tikzpicture}[x=0.75pt,y=0.75pt,yscale=-0.8,xscale=0.8]
%uncomment if require: \path (0,300); %set diagram left start at 0, and has height of 300

%Shape: Circle [id:dp1814086505651734] 
\draw  [fill={rgb, 255:red, 189; green, 16; blue, 224 }  ,fill opacity=1 ] (222.33,103.67) .. controls (222.33,97.22) and (227.56,92) .. (234,92) .. controls (240.44,92) and (245.67,97.22) .. (245.67,103.67) .. controls (245.67,110.11) and (240.44,115.33) .. (234,115.33) .. controls (227.56,115.33) and (222.33,110.11) .. (222.33,103.67) -- cycle ;
%Straight Lines [id:da49306494878637575] 
\draw    (123.67,207.67) -- (284,208) ;
%Straight Lines [id:da508944992470395] 
\draw    (284,131) -- (284,208) ;
%Straight Lines [id:da6580949372672364] 
\draw [color={rgb, 255:red, 189; green, 16; blue, 224 }  ,draw opacity=1 ]   (228.33,115) -- (180.61,204.57) ;
\draw [shift={(179.67,206.33)}, rotate = 298.05] [color={rgb, 255:red, 189; green, 16; blue, 224 }  ,draw opacity=1 ][line width=0.75]    (10.93,-3.29) .. controls (6.95,-1.4) and (3.31,-0.3) .. (0,0) .. controls (3.31,0.3) and (6.95,1.4) .. (10.93,3.29)   ;
%Straight Lines [id:da7980271407074738] 
\draw [color={rgb, 255:red, 189; green, 16; blue, 224 }  ,draw opacity=1 ]   (179.67,206.33) -- (120.46,151.03) ;
\draw [shift={(119,149.67)}, rotate = 43.05] [color={rgb, 255:red, 189; green, 16; blue, 224 }  ,draw opacity=1 ][line width=0.75]    (10.93,-3.29) .. controls (6.95,-1.4) and (3.31,-0.3) .. (0,0) .. controls (3.31,0.3) and (6.95,1.4) .. (10.93,3.29)   ;
%Flowchart: Summing Junction [id:dp38620358199132154] 
\draw  [color={rgb, 255:red, 74; green, 144; blue, 226 }  ,draw opacity=1 ] (94.5,129.33) .. controls (94.5,124.45) and (98.45,120.5) .. (103.33,120.5) .. controls (108.21,120.5) and (112.17,124.45) .. (112.17,129.33) .. controls (112.17,134.21) and (108.21,138.17) .. (103.33,138.17) .. controls (98.45,138.17) and (94.5,134.21) .. (94.5,129.33) -- cycle ; \draw  [color={rgb, 255:red, 74; green, 144; blue, 226 }  ,draw opacity=1 ] (97.09,123.09) -- (109.58,135.58) ; \draw  [color={rgb, 255:red, 74; green, 144; blue, 226 }  ,draw opacity=1 ] (109.58,123.09) -- (97.09,135.58) ;
%Shape: Circle [id:dp2536678430307935] 
\draw  [fill={rgb, 255:red, 0; green, 0; blue, 0 }  ,fill opacity=1 ] (175.83,206.33) .. controls (175.83,204.22) and (177.55,202.5) .. (179.67,202.5) .. controls (181.78,202.5) and (183.5,204.22) .. (183.5,206.33) .. controls (183.5,208.45) and (181.78,210.17) .. (179.67,210.17) .. controls (177.55,210.17) and (175.83,208.45) .. (175.83,206.33) -- cycle ;
%Shape: Circle [id:dp8980388051701753] 
\draw  [fill={rgb, 255:red, 189; green, 16; blue, 224 }  ,fill opacity=1 ] (506.33,103.67) .. controls (506.33,97.22) and (511.56,92) .. (518,92) .. controls (524.44,92) and (529.67,97.22) .. (529.67,103.67) .. controls (529.67,110.11) and (524.44,115.33) .. (518,115.33) .. controls (511.56,115.33) and (506.33,110.11) .. (506.33,103.67) -- cycle ;
%Straight Lines [id:da09191449603737012] 
\draw    (407.67,207.67) -- (568,208) ;
%Straight Lines [id:da786638815474474] 
\draw    (568,131) -- (568,208) ;
%Straight Lines [id:da891177684944767] 
\draw [color={rgb, 255:red, 189; green, 16; blue, 224 }  ,draw opacity=1 ]   (512.33,115) -- (464.61,204.57) ;
\draw [shift={(463.67,206.33)}, rotate = 298.05] [color={rgb, 255:red, 189; green, 16; blue, 224 }  ,draw opacity=1 ][line width=0.75]    (10.93,-3.29) .. controls (6.95,-1.4) and (3.31,-0.3) .. (0,0) .. controls (3.31,0.3) and (6.95,1.4) .. (10.93,3.29)   ;
%Straight Lines [id:da6342734502153051] 
\draw [color={rgb, 255:red, 189; green, 16; blue, 224 }  ,draw opacity=1 ]   (463.67,206.33) -- (405.61,148.02) ;
\draw [shift={(404.2,146.6)}, rotate = 45.13] [color={rgb, 255:red, 189; green, 16; blue, 224 }  ,draw opacity=1 ][line width=0.75]    (10.93,-3.29) .. controls (6.95,-1.4) and (3.31,-0.3) .. (0,0) .. controls (3.31,0.3) and (6.95,1.4) .. (10.93,3.29)   ;
%Flowchart: Summing Junction [id:dp6167414702064523] 
\draw  [color={rgb, 255:red, 74; green, 144; blue, 226 }  ,draw opacity=1 ] (378.5,129.33) .. controls (378.5,124.45) and (382.45,120.5) .. (387.33,120.5) .. controls (392.21,120.5) and (396.17,124.45) .. (396.17,129.33) .. controls (396.17,134.21) and (392.21,138.17) .. (387.33,138.17) .. controls (382.45,138.17) and (378.5,134.21) .. (378.5,129.33) -- cycle ; \draw  [color={rgb, 255:red, 74; green, 144; blue, 226 }  ,draw opacity=1 ] (381.09,123.09) -- (393.58,135.58) ; \draw  [color={rgb, 255:red, 74; green, 144; blue, 226 }  ,draw opacity=1 ] (393.58,123.09) -- (381.09,135.58) ;
%Shape: Circle [id:dp059366421183071894] 
\draw  [fill={rgb, 255:red, 0; green, 0; blue, 0 }  ,fill opacity=1 ] (459.83,206.33) .. controls (459.83,204.22) and (461.55,202.5) .. (463.67,202.5) .. controls (465.78,202.5) and (467.5,204.22) .. (467.5,206.33) .. controls (467.5,208.45) and (465.78,210.17) .. (463.67,210.17) .. controls (461.55,210.17) and (459.83,208.45) .. (459.83,206.33) -- cycle ;
%Straight Lines [id:da29001588556918767] 
\draw [color={rgb, 255:red, 189; green, 16; blue, 224 }  ,draw opacity=1 ]   (529.8,115.4) -- (561.7,152.68) ;
\draw [shift={(563,154.2)}, rotate = 229.45] [color={rgb, 255:red, 189; green, 16; blue, 224 }  ,draw opacity=1 ][line width=0.75]    (10.93,-3.29) .. controls (6.95,-1.4) and (3.31,-0.3) .. (0,0) .. controls (3.31,0.3) and (6.95,1.4) .. (10.93,3.29)   ;
%Straight Lines [id:da6007682340662435] 
\draw [color={rgb, 255:red, 189; green, 16; blue, 224 }  ,draw opacity=1 ]   (566.07,160) -- (475.61,202.55) ;
\draw [shift={(473.8,203.4)}, rotate = 334.81] [color={rgb, 255:red, 189; green, 16; blue, 224 }  ,draw opacity=1 ][line width=0.75]    (10.93,-3.29) .. controls (6.95,-1.4) and (3.31,-0.3) .. (0,0) .. controls (3.31,0.3) and (6.95,1.4) .. (10.93,3.29)   ;
%Straight Lines [id:da41643992801659746] 
\draw [color={rgb, 255:red, 189; green, 16; blue, 224 }  ,draw opacity=1 ]   (566.87,164) -- (548.39,183.16) ;
\draw [shift={(547,184.6)}, rotate = 313.96] [color={rgb, 255:red, 189; green, 16; blue, 224 }  ,draw opacity=1 ][line width=0.75]    (10.93,-3.29) .. controls (6.95,-1.4) and (3.31,-0.3) .. (0,0) .. controls (3.31,0.3) and (6.95,1.4) .. (10.93,3.29)   ;
%Straight Lines [id:da17417546939250383] 
\draw [color={rgb, 255:red, 189; green, 16; blue, 224 }  ,draw opacity=1 ]   (565.27,156.4) -- (543,156.95) ;
\draw [shift={(541,157)}, rotate = 358.58] [color={rgb, 255:red, 189; green, 16; blue, 224 }  ,draw opacity=1 ][line width=0.75]    (10.93,-3.29) .. controls (6.95,-1.4) and (3.31,-0.3) .. (0,0) .. controls (3.31,0.3) and (6.95,1.4) .. (10.93,3.29)   ;
%Straight Lines [id:da8824450616624485] 
\draw [color={rgb, 255:red, 189; green, 16; blue, 224 }  ,draw opacity=1 ]   (455.67,204.4) -- (436.44,196.19) ;
\draw [shift={(434.6,195.4)}, rotate = 23.13] [color={rgb, 255:red, 189; green, 16; blue, 224 }  ,draw opacity=1 ][line width=0.75]    (10.93,-3.29) .. controls (6.95,-1.4) and (3.31,-0.3) .. (0,0) .. controls (3.31,0.3) and (6.95,1.4) .. (10.93,3.29)   ;
%Straight Lines [id:da23937369321783553] 
\draw [color={rgb, 255:red, 189; green, 16; blue, 224 }  ,draw opacity=1 ]   (460.6,197) -- (456.96,177.37) ;
\draw [shift={(456.6,175.4)}, rotate = 79.51] [color={rgb, 255:red, 189; green, 16; blue, 224 }  ,draw opacity=1 ][line width=0.75]    (10.93,-3.29) .. controls (6.95,-1.4) and (3.31,-0.3) .. (0,0) .. controls (3.31,0.3) and (6.95,1.4) .. (10.93,3.29)   ;
%Straight Lines [id:da26822888939006373] 
\draw [color={rgb, 255:red, 189; green, 16; blue, 224 }  ,draw opacity=1 ]   (171.27,204) -- (144.87,194.1) ;
\draw [shift={(143,193.4)}, rotate = 20.56] [color={rgb, 255:red, 189; green, 16; blue, 224 }  ,draw opacity=1 ][line width=0.75]    (10.93,-3.29) .. controls (6.95,-1.4) and (3.31,-0.3) .. (0,0) .. controls (3.31,0.3) and (6.95,1.4) .. (10.93,3.29)   ;
%Straight Lines [id:da06196431043140105] 
\draw [color={rgb, 255:red, 189; green, 16; blue, 224 }  ,draw opacity=1 ]   (177.4,199.4) -- (169.62,175.7) ;
\draw [shift={(169,173.8)}, rotate = 71.83] [color={rgb, 255:red, 189; green, 16; blue, 224 }  ,draw opacity=1 ][line width=0.75]    (10.93,-3.29) .. controls (6.95,-1.4) and (3.31,-0.3) .. (0,0) .. controls (3.31,0.3) and (6.95,1.4) .. (10.93,3.29)   ;
%Straight Lines [id:da4196232868407055] 
\draw [color={rgb, 255:red, 189; green, 16; blue, 224 }  ,draw opacity=1 ]   (189.27,203.6) -- (204.63,191.44) ;
\draw [shift={(206.2,190.2)}, rotate = 141.64] [color={rgb, 255:red, 189; green, 16; blue, 224 }  ,draw opacity=1 ][line width=0.75]    (10.93,-3.29) .. controls (6.95,-1.4) and (3.31,-0.3) .. (0,0) .. controls (3.31,0.3) and (6.95,1.4) .. (10.93,3.29)   ;

% Text Node
\draw (177.75,216.96) node   [align=left] {\begin{minipage}[lt]{8.67pt}\setlength\topsep{0pt}
$\displaystyle P$
\end{minipage}};
% Text Node
\draw (89.25,98.5) node   [align=left] {\begin{minipage}[lt]{8.67pt}\setlength\topsep{0pt}
Ull\\
\end{minipage}};
% Text Node
\draw (229.25,73.5) node   [align=left] {\begin{minipage}[lt]{8.67pt}\setlength\topsep{0pt}
Llum\\\\
\end{minipage}};
% Text Node
\draw (461.75,216.96) node   [align=left] {\begin{minipage}[lt]{8.67pt}\setlength\topsep{0pt}
$\displaystyle P$
\end{minipage}};
% Text Node
\draw (373.25,98.5) node   [align=left] {\begin{minipage}[lt]{8.67pt}\setlength\topsep{0pt}
Ull\\
\end{minipage}};
% Text Node
\draw (513.25,73.5) node   [align=left] {\begin{minipage}[lt]{8.67pt}\setlength\topsep{0pt}
Llum\\\\
\end{minipage}};


\end{tikzpicture}
   \label{fig:indirectLight}
   \caption{representació de la il·luminació directa i indirecta}   


\end{figure}


El metode de Monte Carlo ens ajuda per solucionar aquest problema. Aquest metode 
consisteix en crear $N$ vectors aleatoris $D_i$ continguts en la semiesfera de centre $P$ i radi 1 orientada amb el vector normal
de la superfície, i agafar la llum procedent del raig $raig(P,D_i)$. La quantitat de llum que rebi serà la mitjana dels $N$ raigs. 
\begin{figure}[h]
  \centering 
  

\tikzset{every picture/.style={line width=0.75pt}} %set default line width to 0.75pt        

\begin{tikzpicture}[x=0.75pt,y=0.75pt,yscale=-0.6,xscale=0.6]
%uncomment if require: \path (0,300); %set diagram left start at 0, and has height of 300

%Straight Lines [id:da6890188656157253] 
\draw    (174,212) -- (410,212) ;
%Shape: Circle [id:dp7752522051587584] 
\draw  [fill={rgb, 255:red, 0; green, 0; blue, 0 }  ,fill opacity=1 ] (286,212) .. controls (286,208.69) and (288.69,206) .. (292,206) .. controls (295.31,206) and (298,208.69) .. (298,212) .. controls (298,215.31) and (295.31,218) .. (292,218) .. controls (288.69,218) and (286,215.31) .. (286,212) -- cycle ;
%Straight Lines [id:da2931543775558012] 
\draw    (292,212) -- (252.32,194.07) ;
\draw [shift={(250.5,193.25)}, rotate = 24.31] [color={rgb, 255:red, 0; green, 0; blue, 0 }  ][line width=0.75]    (10.93,-3.29) .. controls (6.95,-1.4) and (3.31,-0.3) .. (0,0) .. controls (3.31,0.3) and (6.95,1.4) .. (10.93,3.29)   ;
%Straight Lines [id:da3647409515441734] 
\draw    (292,212) -- (285.31,169.73) ;
\draw [shift={(285,167.75)}, rotate = 81.01] [color={rgb, 255:red, 0; green, 0; blue, 0 }  ][line width=0.75]    (10.93,-3.29) .. controls (6.95,-1.4) and (3.31,-0.3) .. (0,0) .. controls (3.31,0.3) and (6.95,1.4) .. (10.93,3.29)   ;
%Straight Lines [id:da0431625762330351] 
\draw    (292,212) -- (314.47,174.47) ;
\draw [shift={(315.5,172.75)}, rotate = 120.91] [color={rgb, 255:red, 0; green, 0; blue, 0 }  ][line width=0.75]    (10.93,-3.29) .. controls (6.95,-1.4) and (3.31,-0.3) .. (0,0) .. controls (3.31,0.3) and (6.95,1.4) .. (10.93,3.29)   ;
%Straight Lines [id:da7674777160803408] 
\draw    (292,212) -- (328.78,190.27) ;
\draw [shift={(330.5,189.25)}, rotate = 149.42] [color={rgb, 255:red, 0; green, 0; blue, 0 }  ][line width=0.75]    (10.93,-3.29) .. controls (6.95,-1.4) and (3.31,-0.3) .. (0,0) .. controls (3.31,0.3) and (6.95,1.4) .. (10.93,3.29)   ;
%Shape: Square [id:dp07676771900012436] 
\draw  [fill={rgb, 255:red, 74; green, 144; blue, 226 }  ,fill opacity=1 ] (260,63.25) -- (301.75,63.25) -- (301.75,105) -- (260,105) -- cycle ;
%Shape: Rectangle [id:dp8414726945161215] 
\draw  [fill={rgb, 255:red, 126; green, 211; blue, 33 }  ,fill opacity=1 ] (366,116) -- (436,116) -- (436,147.5) -- (366,147.5) -- cycle ;
%Shape: Rectangle [id:dp31770256724975443] 
\draw  [fill={rgb, 255:red, 208; green, 2; blue, 27 }  ,fill opacity=1 ] (144,115.5) -- (177,115.5) -- (177,175.5) -- (144,175.5) -- cycle ;
%Shape: Arc [id:dp7689365823036822] 
\draw  [draw opacity=0][line width=1.5]  (244.9,212.39) .. controls (244.9,212.18) and (244.9,211.97) .. (244.9,211.76) .. controls (245.03,185.75) and (266.22,164.77) .. (292.24,164.9) .. controls (318.25,165.03) and (339.23,186.22) .. (339.1,212.24) .. controls (339.1,212.29) and (339.1,212.34) .. (339.1,212.39) -- (292,212) -- cycle ; \draw  [color={rgb, 255:red, 248; green, 231; blue, 28 }  ,draw opacity=1 ][line width=1.5]  (244.9,212.39) .. controls (244.9,212.18) and (244.9,211.97) .. (244.9,211.76) .. controls (245.03,185.75) and (266.22,164.77) .. (292.24,164.9) .. controls (318.25,165.03) and (339.23,186.22) .. (339.1,212.24) .. controls (339.1,212.29) and (339.1,212.34) .. (339.1,212.39) ;
%Straight Lines [id:da9807992926893758] 
\draw    (292,212) -- (262.87,181.2) ;
\draw [shift={(261.5,179.75)}, rotate = 46.6] [color={rgb, 255:red, 0; green, 0; blue, 0 }  ][line width=0.75]    (10.93,-3.29) .. controls (6.95,-1.4) and (3.31,-0.3) .. (0,0) .. controls (3.31,0.3) and (6.95,1.4) .. (10.93,3.29)   ;
%Straight Lines [id:da281069296683675] 
\draw [color={rgb, 255:red, 208; green, 2; blue, 27 }  ,draw opacity=1 ] [dash pattern={on 0.84pt off 2.51pt}]  (180,161.25) -- (250.5,193.25) ;
%Straight Lines [id:da7884769291248609] 
\draw [color={rgb, 255:red, 0; green, 0; blue, 0 }  ,draw opacity=1 ] [dash pattern={on 0.84pt off 2.51pt}]  (190.5,95.25) -- (261.5,179.75) ;
%Straight Lines [id:da8625625821418754] 
\draw [color={rgb, 255:red, 74; green, 144; blue, 226 }  ,draw opacity=1 ] [dash pattern={on 0.84pt off 2.51pt}]  (275,106.75) -- (285,167.75) ;
%Straight Lines [id:da7177621326669087] 
\draw [color={rgb, 255:red, 0; green, 0; blue, 0 }  ,draw opacity=1 ] [dash pattern={on 0.84pt off 2.51pt}]  (366,100.75) -- (315.5,172.75) ;
%Straight Lines [id:da6037482218996579] 
\draw [color={rgb, 255:red, 126; green, 211; blue, 33 }  ,draw opacity=1 ] [dash pattern={on 0.84pt off 2.51pt}]  (390.5,149.25) -- (330.5,189.25) ;

\end{tikzpicture}
  \label{fig:randDirections}
  \caption{represerntació del metode de monteCarlo}
\end{figure}


\begin{equation}
   Llum_P= \frac{1}{N}\sum_{i=0}^{N} raig(P, D_i)
\end{equation}



Podem aplicar el Metode de MonteCarlo també a l'hora de generar el raig que surt de la camera i que passa pel píxel. Si només
escollim el raig que passa pel centre del píxel per pintar tot el píxel és possible que no sigui el millor color per representar
l'escena, i obtinguem les vores de objectes següents: 
%diagrama.


\textbf{IMATGE vores PIXELAdes} 



Per evitar aixó podem generar $N$ raigs que surtin de la camera i que passin per punts diferents del píxel de la quadrícula. El
color d'aquest píxel serà
\begin{equation}
   ColorPixel_{ij}= \sum_{s=0}^N colorRaig(Camera,r^s_{ij})
\end{equation}

\begin{figure}[h]
  \centering


\tikzset{every picture/.style={line width=0.75pt}} %set default line width to 0.75pt

\begin{tikzpicture}[x=0.75pt,y=0.75pt,yscale=-0.8,xscale=0.8]
%uncomment if require: \path (0,300); %set diagram left start at 0, and has height of 300

%Flowchart: Summing Junction [id:dp13459297560154682]
\draw  [color={rgb, 255:red, 74; green, 144; blue, 226 }  ,draw opacity=1 ][line width=1.5]  (167.33,97.92) .. controls (167.33,93.27) and (171.1,89.5) .. (175.75,89.5) .. controls (180.4,89.5) and (184.17,93.27) .. (184.17,97.92) .. controls (184.17,102.57) and (180.4,106.33) .. (175.75,106.33) .. controls (171.1,106.33) and (167.33,102.57) .. (167.33,97.92) -- cycle ; \draw  [color={rgb, 255:red, 74; green, 144; blue, 226 }  ,draw opacity=1 ][line width=1.5]  (169.8,91.97) -- (181.7,103.87) ; \draw  [color={rgb, 255:red, 74; green, 144; blue, 226 }  ,draw opacity=1 ][line width=1.5]  (181.7,91.97) -- (169.8,103.87) ;
%Shape: Parallelogram [id:dp2996981126747642]
\draw   (339.75,82.22) -- (339.76,124.24) -- (298.01,106.24) -- (298,64.22) -- cycle ;
%Shape: Circle [id:dp7009304958412188]
\draw  [color={rgb, 255:red, 208; green, 2; blue, 27 }  ,draw opacity=1 ] (319.63,94.75) .. controls (319.63,90.95) and (322.7,87.88) .. (326.5,87.88) .. controls (330.3,87.88) and (333.38,90.95) .. (333.38,94.75) .. controls (333.38,98.55) and (330.3,101.63) .. (326.5,101.63) .. controls (322.7,101.63) and (319.63,98.55) .. (319.63,94.75) -- cycle ;
%Shape: Circle [id:dp41011024141235985]
\draw  [color={rgb, 255:red, 208; green, 2; blue, 27 }  ,draw opacity=1 ] (307.63,106.75) .. controls (307.63,102.95) and (310.7,99.88) .. (314.5,99.88) .. controls (318.3,99.88) and (321.38,102.95) .. (321.38,106.75) .. controls (321.38,110.55) and (318.3,113.63) .. (314.5,113.63) .. controls (310.7,113.63) and (307.63,110.55) .. (307.63,106.75) -- cycle ;
%Shape: Circle [id:dp27116879288931406]
\draw  [color={rgb, 255:red, 208; green, 2; blue, 27 }  ,draw opacity=1 ] (304.81,82.75) .. controls (304.81,79.89) and (307.14,77.56) .. (310,77.56) .. controls (312.86,77.56) and (315.19,79.89) .. (315.19,82.75) .. controls (315.19,85.61) and (312.86,87.94) .. (310,87.94) .. controls (307.14,87.94) and (304.81,85.61) .. (304.81,82.75) -- cycle ;
%Straight Lines [id:da8217561400187317]
\draw    (192,94.25) -- (308.01,82.94) ;
\draw [shift={(310,82.75)}, rotate = 174.43] [color={rgb, 255:red, 0; green, 0; blue, 0 }  ][line width=0.75]    (10.93,-3.29) .. controls (6.95,-1.4) and (3.31,-0.3) .. (0,0) .. controls (3.31,0.3) and (6.95,1.4) .. (10.93,3.29)   ;
%Straight Lines [id:da01695358232869304]
\draw    (195,97.75) -- (324.5,94.8) ;
\draw [shift={(326.5,94.75)}, rotate = 178.69] [color={rgb, 255:red, 0; green, 0; blue, 0 }  ][line width=0.75]    (10.93,-3.29) .. controls (6.95,-1.4) and (3.31,-0.3) .. (0,0) .. controls (3.31,0.3) and (6.95,1.4) .. (10.93,3.29)   ;
%Straight Lines [id:da30679280241900864]
\draw    (196,104.25) -- (312.5,106.71) ;
\draw [shift={(314.5,106.75)}, rotate = 181.21] [color={rgb, 255:red, 0; green, 0; blue, 0 }  ][line width=0.75]    (10.93,-3.29) .. controls (6.95,-1.4) and (3.31,-0.3) .. (0,0) .. controls (3.31,0.3) and (6.95,1.4) .. (10.93,3.29)   ;

% Text Node
\draw (170.75,73.63) node   [align=left] {\begin{minipage}[lt]{37.06pt}\setlength\topsep{0pt}
Camera
\end{minipage}};
% Text Node
\draw (330.75,57.63) node   [align=left] {\begin{minipage}[lt]{37.06pt}\setlength\topsep{0pt}
Píxel
\end{minipage}};
% Text Node
\draw (245.75,115.63) node   [align=left] {\begin{minipage}[lt]{15.3pt}\setlength\topsep{0pt}
r\_i
\end{minipage}};


\end{tikzpicture}
  \label{fig:montecarloCamera}
  \caption{Metode de Monte Carlo aplicat al generador de raigs}
\end{figure}



\section{Materials}
\subsection{Rugositat}
\subsection{Reflecció i refracció}
la refracció és el canvi en la direcció d'una ona que passa d'un medi a un altre. Aquest fenomen segueix la llei de Snell.
\begin{theorem}(Llei d'Snell)
   La llei d'Snell estipula que la raó de l'angle d'incidència amb l'angle de refracció és igual a la raó de les velocitats de
   l'ona pels diferents  medis 
(o la raó dels index de refracció).
\end{theorem}
L'index de refracció està definit com
\begin{equation*}
   n=\frac{c}{v}
\end{equation*}
on $c$ és la constant de la velocitat de la llum al buit i $v$ és la velocitat de la llum en el medi.
\begin{equation}
   \frac{\sin(\theta_1)}{\sin(\theta_2)} =\frac{v_1}{v_2}= \frac{n_2}{n_1}
\end{equation}


\begin{proof}
   Considerem dos raigs paral·lels $A$ i $B$ que passen del medi 1 al medi 2. Al arribar a la frontera segueixen el seu curs
   paral·lels amb direccions $U$ i  $V$ respectivament. La velocitat al medi 1 és $v_1=\frac{c}{n_1}$ i al medi 2 $v_2=\frac{c}{n_2}$.
   Suposem que a temps  $T=0$ el raig  $A$ intersecta amb la frontera al punt  $C$, i en aquest mateix instant, el raig  $B$ es
   troba al punt  $P$.

   $B$ viatge a velocitat  $v_1$ i arriba al punt $D$ en  $t$ segons. Durant aquest interval de temps, el raig $A$ ha viatjat a
   velocitat  $v_2$ i en temps $T=t$ es troba al punt  $Q$.
   Veiem la figura: XXXXXX per veure la situació.
   \begin{figure}[h]
      \centering
\tikzset{every picture/.style={line width=0.75pt}} %set default line width to 0.75pt        

\begin{tikzpicture}[x=0.75pt,y=0.75pt,yscale=-1,xscale=1]
%uncomment if require: \path (0,300); %set diagram left start at 0, and has height of 300

%Straight Lines [id:da45689473753689047] 
\draw    (249.4,163) -- (501.03,163) ;
%Straight Lines [id:da9018058203628643] 
\draw [color={rgb, 255:red, 189; green, 16; blue, 224 }  ,draw opacity=1 ]   (260,95.5) -- (326.09,161.59) ;
\draw [shift={(327.5,163)}, rotate = 225] [color={rgb, 255:red, 189; green, 16; blue, 224 }  ,draw opacity=1 ][line width=0.75]    (10.93,-3.29) .. controls (6.95,-1.4) and (3.31,-0.3) .. (0,0) .. controls (3.31,0.3) and (6.95,1.4) .. (10.93,3.29)   ;
%Straight Lines [id:da5642989616739449] 
\draw    (327.5,87.5) -- (327.5,163) ;
%Shape: Arc [id:dp18652508134029044] 
\draw  [draw opacity=0] (306.26,142.5) .. controls (311.45,136.23) and (319,132.28) .. (327.4,132.25) -- (327.5,163) -- cycle ; \draw   (306.26,142.5) .. controls (311.45,136.23) and (319,132.28) .. (327.4,132.25) ;
%Straight Lines [id:da9106559991278875] 
\draw [color={rgb, 255:red, 74; green, 144; blue, 226 }  ,draw opacity=1 ]   (327.5,163) -- (358.6,233.97) ;
\draw [shift={(359.4,235.8)}, rotate = 246.34] [color={rgb, 255:red, 74; green, 144; blue, 226 }  ,draw opacity=1 ][line width=0.75]    (10.93,-3.29) .. controls (6.95,-1.4) and (3.31,-0.3) .. (0,0) .. controls (3.31,0.3) and (6.95,1.4) .. (10.93,3.29)   ;
%Straight Lines [id:da2842142755891415] 
\draw    (327.5,163) -- (327.5,238.5) ;
%Shape: Arc [id:dp6567010732445318] 
\draw  [draw opacity=0] (339.52,190.89) .. controls (335.89,192.71) and (331.85,193.74) .. (327.59,193.75) -- (327.5,163) -- cycle ; \draw   (339.52,190.89) .. controls (335.89,192.71) and (331.85,193.74) .. (327.59,193.75) ;
%Straight Lines [id:da390884412234318] 
\draw [color={rgb, 255:red, 189; green, 16; blue, 224 }  ,draw opacity=1 ]   (349,95.5) -- (415.09,161.59) ;
\draw [shift={(416.5,163)}, rotate = 225] [color={rgb, 255:red, 189; green, 16; blue, 224 }  ,draw opacity=1 ][line width=0.75]    (10.93,-3.29) .. controls (6.95,-1.4) and (3.31,-0.3) .. (0,0) .. controls (3.31,0.3) and (6.95,1.4) .. (10.93,3.29)   ;
%Straight Lines [id:da10603245404125117] 
\draw    (416.5,87.5) -- (416.5,163) ;
%Straight Lines [id:da507896653263994] 
\draw [color={rgb, 255:red, 74; green, 144; blue, 226 }  ,draw opacity=1 ]   (416.5,163) -- (447.6,233.97) ;
\draw [shift={(448.4,235.8)}, rotate = 246.34] [color={rgb, 255:red, 74; green, 144; blue, 226 }  ,draw opacity=1 ][line width=0.75]    (10.93,-3.29) .. controls (6.95,-1.4) and (3.31,-0.3) .. (0,0) .. controls (3.31,0.3) and (6.95,1.4) .. (10.93,3.29)   ;
%Straight Lines [id:da736965053315968] 
\draw    (416.5,163) -- (416.5,238.5) ;
%Straight Lines [id:da11484595264112962] 
\draw [color={rgb, 255:red, 189; green, 16; blue, 224 }  ,draw opacity=1 ]   (327.5,163) -- (371.35,119.15) ;
%Shape: Arc [id:dp9198774677315408] 
\draw  [draw opacity=0] (348.76,142.52) .. controls (353.19,147.87) and (355.92,154.91) .. (356,162.64) -- (327.5,163) -- cycle ; \draw   (348.76,142.52) .. controls (353.19,147.87) and (355.92,154.91) .. (356,162.64) ;
%Straight Lines [id:da510362388001658] 
\draw [color={rgb, 255:red, 74; green, 144; blue, 226 }  ,draw opacity=1 ]   (343.45,199.4) -- (416.5,163) ;
%Shape: Arc [id:dp7338203790483613] 
\draw  [draw opacity=0] (390.35,174.83) .. controls (388.87,171.29) and (388.04,167.39) .. (388,163.28) -- (416.5,163) -- cycle ; \draw   (390.35,174.83) .. controls (388.87,171.29) and (388.04,167.39) .. (388,163.28) ;
%Shape: Circle [id:dp5949839561647454] 
\draw  [fill={rgb, 255:red, 0; green, 0; blue, 0 }  ,fill opacity=1 ] (323.81,163) .. controls (323.81,160.96) and (325.46,159.31) .. (327.5,159.31) .. controls (329.54,159.31) and (331.19,160.96) .. (331.19,163) .. controls (331.19,165.04) and (329.54,166.69) .. (327.5,166.69) .. controls (325.46,166.69) and (323.81,165.04) .. (323.81,163) -- cycle ;
%Shape: Circle [id:dp07228918015209318] 
\draw  [fill={rgb, 255:red, 0; green, 0; blue, 0 }  ,fill opacity=1 ] (412.81,163) .. controls (412.81,160.96) and (414.46,159.31) .. (416.5,159.31) .. controls (418.54,159.31) and (420.19,160.96) .. (420.19,163) .. controls (420.19,165.04) and (418.54,166.69) .. (416.5,166.69) .. controls (414.46,166.69) and (412.81,165.04) .. (412.81,163) -- cycle ;
%Shape: Circle [id:dp27078225745195517] 
\draw  [fill={rgb, 255:red, 0; green, 0; blue, 0 }  ,fill opacity=1 ] (367.66,119.15) .. controls (367.66,117.11) and (369.31,115.46) .. (371.35,115.46) .. controls (373.39,115.46) and (375.04,117.11) .. (375.04,119.15) .. controls (375.04,121.19) and (373.39,122.84) .. (371.35,122.84) .. controls (369.31,122.84) and (367.66,121.19) .. (367.66,119.15) -- cycle ;
%Shape: Circle [id:dp8329490974829746] 
\draw  [fill={rgb, 255:red, 0; green, 0; blue, 0 }  ,fill opacity=1 ] (339.76,199.4) .. controls (339.76,197.36) and (341.41,195.71) .. (343.45,195.71) .. controls (345.49,195.71) and (347.14,197.36) .. (347.14,199.4) .. controls (347.14,201.44) and (345.49,203.09) .. (343.45,203.09) .. controls (341.41,203.09) and (339.76,201.44) .. (339.76,199.4) -- cycle ;

% Text Node
\draw (312.1,116) node   [align=left] {\begin{minipage}[lt]{15.91pt}\setlength\topsep{0pt}
$\displaystyle \theta _{1}$
\end{minipage}};
% Text Node
\draw (254.7,88.1) node   [align=left] {\begin{minipage}[lt]{15.91pt}\setlength\topsep{0pt}
$\displaystyle A$
\end{minipage}};
% Text Node
\draw (370.7,233.1) node   [align=left] {\begin{minipage}[lt]{15.91pt}\setlength\topsep{0pt}
$\displaystyle U$
\end{minipage}};

% Text Node
\draw (350.2,83.6) node   [align=left] {\begin{minipage}[lt]{15.91pt}\setlength\topsep{0pt}
$\displaystyle B$
\end{minipage}};
% Text Node
\draw (364.1,148) node   [align=left] {\begin{minipage}[lt]{15.91pt}\setlength\topsep{0pt}
$\displaystyle \theta _{1}$
\end{minipage}};
% Text Node
\draw (320.2,168.6) node   [align=left] {\begin{minipage}[lt]{15.91pt}\setlength\topsep{0pt}
$\displaystyle C$
\end{minipage}};
% Text Node
\draw (412.2,172.1) node   [align=left] {\begin{minipage}[lt]{15.91pt}\setlength\topsep{0pt}
$\displaystyle D$
\end{minipage}};
% Text Node
\draw (460.2,230.6) node   [align=left] {\begin{minipage}[lt]{15.91pt}\setlength\topsep{0pt}
$\displaystyle V$
\end{minipage}};
% Text Node
\draw (379.18,168.36) node   [align=left] {\begin{minipage}[lt]{11.85pt}\setlength\topsep{0pt}
$\displaystyle \theta _{2}$
\end{minipage}};
% Text Node
\draw (334.85,198.36) node   [align=left] {\begin{minipage}[lt]{11.85pt}\setlength\topsep{0pt}
$\displaystyle \theta _{2}$
\end{minipage}};
% Text Node
\draw (381.2,108.93) node   [align=left] {\begin{minipage}[lt]{15.91pt}\setlength\topsep{0pt}
$\displaystyle P$
\end{minipage}};
% Text Node
\draw (358.53,201.27) node   [align=left] {\begin{minipage}[lt]{15.91pt}\setlength\topsep{0pt}
$\displaystyle Q$
\end{minipage}};


\end{tikzpicture}
      \label{Snell}
      \caption{Representació de la demostració de la llei d'Snell}
      
   \end{figure}

Ara només cal aplicar simple trigonometria. Sigui $x$ la distància entre  $C$ i  $D$,
 \begin{align*}
    x \sin(\theta_1) = \|PD\| = v_1t= \frac{c}{n_1} t  \\
    x \sin(\theta_2) = \|CQ\| = v_2t= \frac{c}{n_2} t  \\
\end{align*}
per tant 
\begin{align*}
   n_1 \sin(\theta_1) = \frac{c}{x}t \\
   n_2 \sin(\theta_2) = \frac{c}{x}t \\
\end{align*}
D'on podem deduir la Llei d'Snell
  \begin{equation*}
     \frac{\sin(\theta_1)}{\sin(\theta_2)}= \frac{n_2}{n_1}
  \end{equation*} 
\end{proof}


En el nostre cas, tenim que el medi 1 és l'aire i el medi 2 podrà ser aigua, vidre, o el que decidim.
Ara, utilitzant la llei d'Snell haurem de trobar el vector director del raig un cop atravessat el segon medi. Vegem el diagrama
següent per fer els càlculs necessaris.



\begin{figure}[h]
   \centering

\tikzset{every picture/.style={line width=0.75pt}} %set default line width to 0.75pt        

\begin{tikzpicture}[x=0.75pt,y=0.75pt,yscale=-1,xscale=1]
%uncomment if require: \path (0,300); %set diagram left start at 0, and has height of 300

%Straight Lines [id:da23355097040363781] 
\draw    (249.4,163) -- (415,163) ;
%Straight Lines [id:da8264060316446564] 
\draw [color={rgb, 255:red, 189; green, 16; blue, 224 }  ,draw opacity=1 ]   (260,95.5) -- (326.09,161.59) ;
\draw [shift={(327.5,163)}, rotate = 225] [color={rgb, 255:red, 189; green, 16; blue, 224 }  ,draw opacity=1 ][line width=0.75]    (10.93,-3.29) .. controls (6.95,-1.4) and (3.31,-0.3) .. (0,0) .. controls (3.31,0.3) and (6.95,1.4) .. (10.93,3.29)   ;
%Straight Lines [id:da9234880463309213] 
\draw    (327.5,89.5) -- (327.5,163) ;
\draw [shift={(327.5,87.5)}, rotate = 90] [color={rgb, 255:red, 0; green, 0; blue, 0 }  ][line width=0.75]    (10.93,-3.29) .. controls (6.95,-1.4) and (3.31,-0.3) .. (0,0) .. controls (3.31,0.3) and (6.95,1.4) .. (10.93,3.29)   ;
%Shape: Arc [id:dp4137864468174903] 
\draw  [draw opacity=0] (306.26,142.5) .. controls (311.45,136.23) and (319,132.28) .. (327.4,132.25) -- (327.5,163) -- cycle ; \draw   (306.26,142.5) .. controls (311.45,136.23) and (319,132.28) .. (327.4,132.25) ;
%Straight Lines [id:da6909116941952036] 
\draw [color={rgb, 255:red, 189; green, 16; blue, 224 }  ,draw opacity=1 ]   (327.5,163) -- (358.6,233.97) ;
\draw [shift={(359.4,235.8)}, rotate = 246.34] [color={rgb, 255:red, 189; green, 16; blue, 224 }  ,draw opacity=1 ][line width=0.75]    (10.93,-3.29) .. controls (6.95,-1.4) and (3.31,-0.3) .. (0,0) .. controls (3.31,0.3) and (6.95,1.4) .. (10.93,3.29)   ;
%Straight Lines [id:da5674949792410671] 
\draw    (327.5,163) -- (327.5,236.5) ;
\draw [shift={(327.5,238.5)}, rotate = 270] [color={rgb, 255:red, 0; green, 0; blue, 0 }  ][line width=0.75]    (10.93,-3.29) .. controls (6.95,-1.4) and (3.31,-0.3) .. (0,0) .. controls (3.31,0.3) and (6.95,1.4) .. (10.93,3.29)   ;
%Shape: Arc [id:dp8825063244202399] 
\draw  [draw opacity=0] (339.52,190.89) .. controls (335.89,192.71) and (331.85,193.74) .. (327.59,193.75) -- (327.5,163) -- cycle ; \draw   (339.52,190.89) .. controls (335.89,192.71) and (331.85,193.74) .. (327.59,193.75) ;
%Shape: Rectangle [id:dp04228228217536534] 
\draw  [color={rgb, 255:red, 0; green, 0; blue, 0 }  ,draw opacity=0 ][fill={rgb, 255:red, 74; green, 144; blue, 226 }  ,fill opacity=0.26 ] (249.4,163) -- (415,163) -- (415,244.6) -- (249.4,244.6) -- cycle ;

% Text Node
\draw (312.1,116) node   [align=left] {\begin{minipage}[lt]{15.91pt}\setlength\topsep{0pt}
$\displaystyle \theta _{1}$
\end{minipage}};
% Text Node
\draw (343.7,85.6) node   [align=left] {\begin{minipage}[lt]{15.91pt}\setlength\topsep{0pt}
$\displaystyle N$
\end{minipage}};
% Text Node
\draw (340,202.4) node   [align=left] {\begin{minipage}[lt]{15.91pt}\setlength\topsep{0pt}
$\displaystyle \theta _{2}$
\end{minipage}};
% Text Node
\draw (374.1,140.4) node   [align=left] {\begin{minipage}[lt]{15.91pt}\setlength\topsep{0pt}
$\displaystyle n_{1}$
\end{minipage}};
% Text Node
\draw (372.1,178.8) node   [align=left] {\begin{minipage}[lt]{15.91pt}\setlength\topsep{0pt}
$\displaystyle n_{2}$
\end{minipage}};
% Text Node
\draw (239.7,89.6) node   [align=left] {\begin{minipage}[lt]{15.91pt}\setlength\topsep{0pt}
$\displaystyle R$
\end{minipage}};
% Text Node
\draw (373.7,233.6) node   [align=left] {\begin{minipage}[lt]{15.91pt}\setlength\topsep{0pt}
$\displaystyle R'$
\end{minipage}};
% Text Node
\draw (327.7,250.1) node   [align=left] {\begin{minipage}[lt]{15.91pt}\setlength\topsep{0pt}
$\displaystyle N'$
\end{minipage}};


\end{tikzpicture}
   \label{refraccio}
   \caption{diagrama de la refracció d'un raig de llum sobre un material}
   
\end{figure}

Per trobar $R'$ primer el sempararem en el component  paral·lel i perpendicular a $N'$.
\begin{align*}
   R=R_{ \bot }+ R_{ \parallel}\\
   R'=R'_{ \bot }+ R'_{ \parallel}
\end{align*}

com $sin(\theta_2)= \| R'_{ \bot}\|$, es compleix per la llei de Snell
\begin{align*}
   \|R'_{ \bot}\|=\frac{n_1}{n_2} \|R_{ \bot}\|
\end{align*}
i en particular
\begin{align*}
   R'_{ \bot}=\frac{n_1}{n_2} R_{ \bot} %aqui estava multiplicat per -1
\end{align*}

també sabem que 
\begin{align*}
   R_{ \bot} = -R-\cos(\theta_1)N
\end{align*}
%aquest "-" davant de r_1 és perque el vector esta girat
per tant 
\begin{equation*}
   R'_{ \bot} = -\frac{n_1}{n_2} (R+\cos(\theta_1)N)
\end{equation*}
Com ja hem vist a l'apartat de Reflexió podem expressar el cossinus de l'angles entre dos vectors com $\cos(\theta_1)= <-R, N>$.

Per trobar l'expresió de $R'_{\parallel}$ tenim que 
\begin{align*}
   R'_{\parallel} &= \cos(\theta_2)N'\\
                  &= \sqrt{1-\|\sin(\theta_2)\|^2} N' \\ 
                  &= -\sqrt{1-\|R'_{\bot}\|^2}N
\end{align*}
Ja tenim els dos components d'$R'$ en funció d' $R$ i  $N$.


els index de refracció més abituals i que posarem en pràctica són
\begin{itemize}
   \item aire $ \approx 1$
   \item aigua $=4 / 3$
   \item vidre  $=1.52$
\end{itemize}



\section{Objectes no parametritzables}
\subsection{julia set}
\subsection{aproximació del vector normal}
\subsection{quaternions}







\end{document}


